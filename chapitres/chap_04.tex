
\chapter[Programmation]{Implémentation}
    Dans ce chapitre sont présentés les aspects liés à l'implémentation
    informatique de l'application. Seuls les points clefs seront mis en exergue, ainsi 
    une fois présenté un certain aspect, toutes ses apparitions disséminée dans le code
    ne seront pas énumérées. 

    \section{Choix des téchnologies}
    Le développement d'appications
    pour smartphone est depuis un décénie en pleine efervesence et il existe par 
    conséquent une multitude de frameworks, services, langages, méthodologies et paradigmes liés à leur
    développement.

    Ces téchnologies, aux noms éxotiques, aux logos plus brillants les uns que les autres et leurs 
    promotions par diverses conférences, font que leurs différences relèvent plus
    d'une stratégie marketing, visant les informaticiens, réussie que des attributs
    intrinsèques de la téchnologie en question. 

    L'application étant d'une compléxité modérée et ne demandant pas de resource importantes
    comme tel pourrait être le cas pour une messagerie instantanée à grand échelle ou une application
    utilisant abondament un domaine spécifique de connaissance comme le machine learning, le traitement d'images, les jeux vidéos, etc.
    Exclu de facto le choix d'une téchnologie basée exclusivement sur les performances ou sur le développement natif.

    N'ayant jamais fait cela auparavant, il n'y a aucune préférence de ma part pour telle ou telle téchnologie.

    Ces constats donnent lieu aux critères de selections suivants:
    \smallskip
    \begin{itemize}
        \item Développement cross-platteform
        \item Simplicité
        \item Apprentissage d'un langage plutôt qu'une multitude
        \item Vaste documentation et ressources d'apprentissage
    \end{itemize}
    \smallskip
    Le premier critères est celui qui réduit le plus la liste des possibilités. En effet, les frameworks 
    permetant le développement d'applications pour Android et IOS ne se comptent pas en grand nombre. Il existe:
    \smallskip
    \begin{itemize}
        \item Xamarin - Microsoft
        \item React Native - Facebook
        \item Flutter - Google
        \item Adobe PhoneGap - Adobe
        \item Ionic - MIT
    \end{itemize}
    \smallskip
    Le choix parmis ces possibilités découle essentiellement de l'arbitraire. Toutefois, Ionic à été exclu
    car il est nécessaire de maitriser HTML5 et par conséquent CSS mais encore Angular JS. Ce qui contredit le 3ème critère.

    React native a été exclu pour des raisons similaires. I.e. l'apprentissage de divers langage.

    Finalement, suite à un cours de Academind d'une durée de 40 heures portant sur les aspects les
    plus basiques du développement jusqu'au déployement de l'application en passant par le routage, la gestion de requettes http,
    la connexion à tout un écosistème de bases de données, l'utilisation de camera et géolocalisation, et même sur 
    comment changer le logo de l'application, le choix c'est porté sur Flutter.

    Flutter est un framework crée par Google. Ce dernier offre pour les application le service Firebase qui englobe :
    \smallskip
    \begin{itemize}
        \item Cloud Firestore
        \item Real time database
        \item Functions
        \item Machine learning
        \item Cloud messaging
        \item \dots
    \end{itemize}
    \smallskip
    Firebase s'intègre, par conception, particulièrement bien et facilement à Flutter. Même s'ils sont
    indépendants leur utilisation conjointe forme un seul écosistème plus facile à apréhender. Ainsi pour la 
    base de données, le choix à été la Real time database. Car cette dernière fournit une API REST.

    De plus, toujours dans cet ecosystème, Cloud messaging est utilisé pour l'envoit des notifications et Functions pour 
    effectuer des actions côté serveur lorsque la base de donnée subie des modifications. 
    
    \section{Flutter}

    \subsection{Les widgets}

    \subsection{Flux des données}

        \subsubsection{Par constructeur}

        \subsubsection{Provider \& Consumer}


    \section{La base de données}
        \subsection{La méthodologie REST}
        \subsection{REST appliqué à Dart}

            
    \section{Notifications}


    
    \section{Sécurité}

        \subsection{Authentification}

        \subsection{Firebase rules}


\chapter[Programmation]{Implémentation}
    Dans ce chapitre sont présentés les aspects liés à l'implémentation
    informatique de l'application. Seuls les points clefs seront mis en exergue, ainsi 
    une fois présenté un certain aspect, toutes ses apparitions disséminée dans le code
    ne seront pas énumérées. 

    \section{Choix des technologies}
    Le développement d'applications
    pour smartphones est depuis un peu plus d'une décennie en pleine effervescence et il existe, par 
    conséquent, une multitude de frameworks, services, langages, méthodologies et paradigmes liés à leur
    développement.

    Ces technologies, aux noms exotiques, aux logos plus brillants les uns que les autres et leurs 
    promotions par diverses conférences, font que leurs différences relèvent plus
    d'une stratégie marketing, visant les informaticiens, réussie que des attributs
    intrinsèques de la technologie en question. 

    L'application étant d'une complexité modérée et ne demandant pas de ressources importantes
    comme tel pourrait être le cas pour une messagerie instantanée à grand échelle ou une application
    utilisant abondamment un domaine spécifique de connaissance comme le machine learning, le traitement d'images, les jeux vidéos, etc.
    Exclu \textit{de facto} le choix d'une technologie basée exclusivement sur les performances ou sur le développement natif.

    N'ayant jamais fait cela auparavant, il n'y a aucune préférence de ma part pour telle ou telle technologie.

    Ces constats donnent lieu aux critères de sélection suivants:
    \smallskip
    \begin{itemize}
        \item Développement cross-plateforme
        \item Simplicité
        \item Apprentissage d'un langage plutôt qu'une multitude
        \item Vaste documentation et ressources d'apprentissage
    \end{itemize}
    \smallskip
    Le premier critères est celui qui réduit le plus la liste des possibilités. En effet, les frameworks 
    permettant le développement d'applications pour Android et IOS ne se comptent pas en grand nombre. Il existe:
    \smallskip
    \begin{itemize}
        \item Xamarin - Microsoft
        \item React Native - Facebook
        \item Flutter - Google
        \item Adobe PhoneGap - Adobe
        \item Ionic - MIT
    \end{itemize}
    \smallskip
    Le choix parmi ces possibilités découle essentiellement de l'arbitraire. Toutefois, Ionic à été exclu
    car il est nécessaire de maîtriser HTML5 et par conséquent CSS mais encore Angular JS. Ce qui contredit le 3ème critère.

    React native a été exclu pour des raisons similaires. I.e. l'apprentissage de divers langage.

    Finalement, suite à un cours de Academind d'une durée de 40 heures portant sur les aspects les
    plus basiques du développement jusqu'au déploiement de l'application en passant par le routage, la gestion de requêtes http,
    la connexion à tout un écosystème de bases de données, l'utilisation de caméra et géolocalisation, et même sur 
    comment changer le logo de l'application, le choix c'est porté sur Flutter.

    Flutter est un framework crée par Google. Ce dernier offre pour les application le service Firebase qui englobe :
    \smallskip
    \begin{itemize}
        \item Cloud Firestore
        \item Real time database
        \item Functions
        \item Machine learning
        \item Cloud messaging
        \item \dots
    \end{itemize}
    \smallskip
    Firebase s'intègre, par conception, particulièrement bien et facilement à Flutter. Même s'ils sont
    indépendants leur utilisation conjointe forme un seul écosystème plus facile à appréhender. Ainsi pour la 
    base de données, le choix à été la Real time database. Car cette dernière fournit une API REST.

    De plus, toujours dans cet ecosystème, Cloud messaging est utilisé pour l'envoi des notifications et Functions pour 
    effectuer des actions côté serveur lorsque la base de donnée subie des modifications. 

    \section{Flutter}
    Flutter est un software développement kit (SDK) développé par Google permettant de concevoir
    des applications pour plusieurs plateformes. Notament, Android et IOS entre autres avec un seul code source.

    Le framework a une architecture avec plusieurs éléments dont 3 sont pertinents pour ce projet:
    \smallskip
    \begin{itemize}
        \item La plateforme Dart
        \item La bibliothèque Fundation
        \item Des widgets spécifiques
    \end{itemize}
    \smallskip
    \subsection*{Dart}
    Dart est le langage de programmation avec lequel certains éléments du framework on été développé mais il s'agit
    surtout du langage dans lequel on implémente une application en Flutter. 

    C'est un langage orienté objet avec un syntaxe de type C à l'image d'autres langages comme Java ou C++.
    Sa syntaxe est proche du Java. 
    
    Il partage certaines fonctionnalités propres au langages fonctionnels. Notamment, l'inférence de type ou encore
    certaines fonctionnalités comme les map, functors \dots

    En Dart, l'allocation de mémoire est gérée automatiquement par un garbage collector. 

    Voici un exemple minimal d'un hello world en Dart:

    \begin{minted}{dart}
        void main() => print('Hello, World!');
    \end{minted}


    \subsection{Les widgets}

    \subsection{Flux des données}

        \subsubsection{Par constructeur}

        \subsubsection{Provider \& Consumer}


    \section{La base de données}
        \subsection{La méthodologie REST}
        \subsection{REST appliqué à Dart}

            
    \section{Notifications}


    
    \section{Sécurité}

        \subsection{Authentification}

        \subsection{Firebase rules}

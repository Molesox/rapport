
\chapter{Conclusion}
En conclusion, les objectifs fixés on été globalement atteints. L'application est fonctionelle et réelement utilisable. Elle n'a pas de soucis de performance et les données transférées à la base de données sont suffisament petites pour n'engendrer aucun coût pour le restaurant. 

Malheureusement, depuis que j'ai finalisé le code de l'application, tous les restaurants sont férmé du à la pandémie du COVID-19. J'ai été et je suis actuellemt encore dans l'incapacité de tester l'application dans une situation réele. 

Il y a toutefois, une multitude de points à améilorer. Notamment, au niveau de l'architecture. J'ai en effet fait la distinction entre le model et la \textit{view}, or ça aurait été encore mieux de suivre un motif de type Model View Controller. 
Ou encore au niveau de la sécurité. Car, dans l'implémentation actuelle, les droit d'accès, c'est-à-dire, ce qui empêche un utilisateur normal d'être un administrateur sont uniquement gérés dans l'application elle-même en interdisant certains écrans d'être affichés. Il faudrait également ajouter des restriction à certaines parties de la base de données, du côté serveur.

Finalement, pour une toute prémière fois dans le développement d'application je suis très fière de mon résultat.  Malgré ses défauts. J'ai pu mettre en application toute sorte de concepts appris ces trois années de bahchelor. Depuis l'héritage jusqu'à la programmation fonctionelle (il n'y a pas une boucle for dans tout le projet), en passant par les bases de données, les documents semi-structurés (json), les design patterns (composite) ou encore les automates à états pour gérér les privilèges. 
\chapter[Introduction]{Introduction}
\cite{einstein}
La bonne gestion des ressources dans une entreprise est inaliénable à son bon fonctionnement. Le personnel et son horaire de travail l'est, par extension, également. 

Un petit restaurant aurait du mal à investir dans un \textbf{ERP}\footnote{Entreprise Resources Planning} pour gérer l'attribution des horaires de son personnel. De plus, ceux-ci sont conçu pour une certaine stabilité et non pas pour des changements très fréquents.

En collaboration avec le restaurant du Crazy Wolf, à Fribourg, j'ai développé une application pour gérer les horaires et les échanges de services du personnel.

\section[Motivations et objectifs]{Motivation et Objectifs}

\subsection*{Motivation}
J'ai travaillé un an dans le restaurant du Crazy Wolf.

En ce qui concerne les serveurs, il y en a beaucoup. Environ une quinzaine. Pendant cette année de travail j'ai constaté que les serveurs, étant majoritairement étudiants, avaient, pour la plupart, de petits pourcentages d'occupation. Un ou deux services par semaine. Par service en entend la tranche horaire de travail de midi ou du soir. 
De par la jeunesse du personnel et des occupations annexes que les serveurs ou serveuses ont, les échanges sont très fréquents. 

Actuellement le restaurant du Crazy Wolf utilise une messagerie instantanée pour gérer la distribution d'horaires, les échanges entre les serveurs, les imprévus, \dots Et ce, avec les inconvénients qui s'imposent: les demandes de remplacements se perdent dans l'historique de conversation, du personnel oublie de se présenter, le même message et répété plusieurs fois, entre autres aléas.



Ainsi, l'idée d'une gestion centralisée de l'horaire et des échanges de services m'est venu. En discutant avec plusieurs membres du personnel et avec les patrons, ils ont confirmé que cette idée répondait à une problématique réelle. 

\subsection*{Objectifs}

Créer une application disponible sur Android et IOS pour la gestion des horaires et des échanges. L'objectif est mesurable. En effet, si la messagerie instantanée n'est plus ou très peu utilisé alors la création d'une application se justifie.

\newpage

\section[Structure du rapport]{Structure du rapport}

\subsection*{Chapitre I: Introduction}
L'introduction est le discours préliminaire de ce rapport.

\subsection*{Chapitre II: Aspects métier}
Dans ce chapitre se trouve l'analyse de la problématique d'un point de vue métier. C'est-à-dire, les conditions d'utilisation, le type d'utilisateurs, définitions claires d'utilisation. Précisions sur le résultat recherché.

\subsection*{Chapitre III: Présentation de l'application}
Dans ce chapitre l'application est présentée. Son mode de fonctionnement ainsi que des captures d'écran pour décrire plusieurs scénarios d'utilisations possibles.

\subsection*{Chapitre IV: Éléments de programmation}
Dans ce chapitre se situe la partie technique en lien avec l'implémentation. On y retrouve la description du framework utilisé ainsi que les points clefs du développement, au travers d'extraits du code source commentés.

% \subsection*{Chapitre V: Coordination avec le client}
% L'application développée s'est faite en accord avec le restaurant du Crazy Wolf. Dans ce chapitre seront résumés les contacts avec ce dernier ainsi que nos différents échange ou encore les fonctionnalités demandées.

\subsection*{Chapitre V: Conclusion}
Dans ce chapitre seront discutés les résultats du projet. De plus, j'y fais part de mon point de vue personnel.

% \section[Table des symboles]{Conventions d'écriture}

